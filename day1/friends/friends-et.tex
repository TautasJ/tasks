\documentclass{boi2014-et}

\renewcommand{\DayNum}{1}
\renewcommand{\TaskCode}{friends}
\renewcommand{\TaskName}{Kolm sõpra}
\renewcommand{\TaskVersion}{1.3}

\begin{document}

    \begin{wrapfigure}{r}{4cm}
        \vspace{-24pt}
        \includegraphics[width=4cm]{\TaskCode.jpeg}
    \end{wrapfigure}

    Kolm sõpra mängivad järgmist mängu.
    Esimene sõber valib sõne $S$.
    Seejärel moodustab teine sõber uue sõne $T$, mis koosneb sõne $S$ kahekordsest kordusest.
    Lõpuks lisab kolmas sõber ühe tähe sõne $T$ algusse, lõppu või kuhugi keskele, saades nii sõne $U$.

    \Task

    Taastada nii saadud sõne $U$ põhjal esialgne sõne $S$.

    \Input

    Sisendi esimesel real on lõpptulemusena saadud sõne $U$ pikkus $N$ ja teisel real sõne $U$ ise.
    Sõne $U$ koosneb $N$ inglise tähestiku suurtähest (A, B, C, \ldots, Z).

    \Output

    Väljastada esialgne sõne $S$.
    Siiski on kaks erijuhtu:
    \begin{enumerate}
        \item Kui pole võimalik, et sisendis antud sõne $U$ on saadud eelkirjeldatud
            protseduuriga, väljastada tekst {\tt NOT POSSIBLE}.
        \item Kui esialgne sõne $S$ pole üheselt määratud, väljastada tekst {\tt NOT UNIQUE}.
    \end{enumerate}

    \Examples

    \simpleexample{7\newline ABXCABC}{ABC}{}
    \simpleexample{6\newline ABCDEF}{NOT POSSIBLE}{}
    \simpleexample{9\newline ABABABABA}{NOT UNIQUE}{}

    \Scoring

    \begin{description}
        \item[Alamülesanne 1 (35 punkti):] $2 \le N \le 2\,001$.
        \item[Alamülesanne 2 (65 punkti):] $2 \le N \le 2\,000\,001$.
    \end{description}

    \Constraints

    \begin{description}
        \item[Ajalimiit:] 0,5 s.
        \item[Mälulimiit:] 256 MB.
    \end{description}

\end{document}
