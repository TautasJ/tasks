\documentclass{boi2014-lt}

\renewcommand{\DayNum}{1}
\renewcommand{\TaskCode}{friends}
\renewcommand{\TaskName}{Trys draugai}
\renewcommand{\TaskVersion}{1.3}

\begin{document}
    \begin{wrapfigure}{r}{4cm}
        \vspace{-24pt}
		\includegraphics[width=4cm]{\TaskCode.jpeg}
	\end{wrapfigure}
	Trys draugai žaidžia tokį žaidimą:
	Pirmasis draugas sugalvoja tam tikrą simbolių eilutę $S$.
	Tada antrasis draugas sukonstruoja naują simbolių eilutę $T$, kuri susideda
	iš dviejų simbolių eilutės $S$ kopijų.
	Galų gale trečiasis draugas įterpia vieną raidę bet kurioje simbolių eilutės
	$T$ pozicijoje ir taip gauna naują simbolių eilutę $U$.

    \Task
    Duotai simbolių eilutei $U$ rekonstruokite pradinę simbolių eilutę $S$.

    \Input
    Pirmoje eilutėje įrašytas vienas sveikasis skaičius $N$ --- galutinės simbolių
    eilutės $U$ ilgis. Pati simbolių eilutė pateikta antroje įvesties eilutėje.
    Ji sudaryta iš $N$ didžiųjų lotyniškų raidžių A, B, C, \ldots{}, Z.

    \Output
    Jūsų sprendimas turėtų išspausdinti pradinę simbolių eilutę $S$.
    Tačiau gali būti keletas išimčių:
    \begin{enumerate}
        \item Jeigu eilutės $U$ neįmanoma suformuoti anksčiau aprašytu būdu,
        jūsų sprendimas turėtų išspausdinti tekstą {\tt NOT POSSIBLE}.
        \item Jeigu yra daugiau nei viena eilutė $S$, iš kurios galima
        suformuoti eilutę $U$, jūsų sprendimas turėtų išspausdinti tekstą
        {\tt NOT UNIQUE}.
    \end{enumerate}
    

    \Examples

    \simpleexample{7\newline ABXCABC}{ABC}{}
    \simpleexample{6\newline ABCDEF}{NOT POSSIBLE}{}
    \simpleexample{9\newline ABABABABA}{NOT UNIQUE}{}

    \Scoring

    \begin{description}
        \item[Dalinė užduotis nr. 1 (35 taškai):] $1 \le N \le 2001$.
        \item[Dalinė užduotis nr. 2 (65 taškai):] $1 \le N \le 2000001$.
    \end{description}

    \Constraints

    \begin{description}
        \item[Laiko limitas:] 0.5 s.
        \item[Atminties limitas:] 256 MB.
    \end{description}

\end{document}

