\documentclass{boi2014-de}

\renewcommand{\DayNum}{1}
\renewcommand{\TaskCode}{friends}
\renewcommand{\TaskName}{Drei Freunde}

\begin{document}
    \begin{wrapfigure}{r}{4cm}
        \vspace{-24pt}
		\includegraphics[width=4cm]{\TaskCode.jpeg}
	\end{wrapfigure}
    Drei Freunde spielen das folgende Spiel:
    Der erste Freund wählt ein Wort.
    Dann konstruiert der zweite Freund ein neues Wort als Verkettung von zwei Kopien des ersten Wortes.
    Schließlich fügt der dritte Freund irgendwo im Wort einen Buchstaben ein.

    \Task
    Gegeben das letzte, vom dritten Freund konstruierte Wort, sollst du das ursprüngliche, vom ersten Freund gewählte Wort rekonstruieren.

    \Input
    Die erste Zeile beinhaltet die Länge $N$ des vom dritten Freund konstruierten Wortes.
    Das Wort selbst bildet die zweite Zeile, bestehend aus $N$ Großbuchstaben A, B, C, \ldots{} Z.

    \Output
    Dein Programm sollte den vom ersten Freund gewählten String ausgeben.
    Es gibt zwei Ausnahmen:
    \begin{enumerate}
        \item Ist es unmöglich, dass das Wort in der Eingabe nach dem oben beschriebenen Ablauf entstanden ist, gib {\tt NOT POSSIBLE} aus.
        \item Ist dies nicht der Fall, doch lässt sich nicht eindeutig auf das vom ersten Freund gewählte Wort schließen, gib {\tt NOT
        UNIQUE} aus.
    \end{enumerate}
    

    \Examples

    \simpleexample{7\newline ABXCABC}{ABC}{}
    \simpleexample{6\newline ABCDEF}{NOT POSSIBLE}{}
    \simpleexample{9\newline ABABABABA}{NOT UNIQUE}{}

    \Scoring

    \begin{description}
        \item[Teilaufgabe 1 (35 Punkte):] $2 \le N \le 2001$.
        \item[Teilaufgabe 2 (65 Punkte):] $2 \le N \le 2000001$.
    \end{description}

    \Constraints

    \begin{description}
        \item[Zeitlimit:] $0.5$ s.
        \item[Speicherlimit:] $256$ MB.
    \end{description}

\end{document}

