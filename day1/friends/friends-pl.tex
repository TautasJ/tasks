\documentclass{boi2014-pl}

\usepackage{enumitem}
\usepackage{todonotes}

\renewcommand{\DayNum}{1}
\renewcommand{\TaskCode}{friends}
\renewcommand{\TaskName}{Przyjaciele}

\begin{document}
    \begin{wrapfigure}{r}{3cm}
		\includegraphics[width=3cm]{\TaskCode.jpeg}
	\end{wrapfigure}
    Trzech przyjaciół lubi następującą zabawę.
    Pierwszy z nich układa pewien napis.
    Następnie drugi konstruuje kolejny napis, składający się z dwóch identycznych kopii napisu ułożonego przez pierwszego.
    Na koniec, trzeci z przyjaciół dokłada jedną literę do napisu ułożonego przez poprzednika.

    \Task
    Masz dany napis końcowy. Zrekonstruuj napis ułożony przez pierwszego z przyjaciół.

    \Input
    W pierwszym wejścia znajduje się długość końcowego napisu $N$.
    W drugim wierszu znajduje się napis.
    Składa się z wielkich liter A, B, C, \ldots{} Z.

    \Output
    Twój program powinien wypisać początkowy napis.
    Jednakże, są dwa wyjątki:
    \begin{enumerate}
        \item Jeśli nie jest możliwe, by w wyniku zabawy z pewnego napisu początkowego powstał podany napis końcowy, program powinien wypisać {\tt NOT POSSIBLE}.
        \item Jeśli początkowy napis można zrekonstruować na więcej niż jeden sposób, program powinien wypisać {\tt NOT
        UNIQUE}.
    \end{enumerate}
    

    \Examples

    \simpleexample{ABXCABC}{ABC}{}
    \simpleexample{ABCDEF}{NOT POSSIBLE}{}
    \simpleexample{ABABABABA}{NOT UNIQUE}{}

    \Scoring

    \begin{description}
        \item[Podzadanie 1 (30 punktów):] Długość napisu końcowego to co najwyżej $2\,001$ znaków.
        \item[Podzadanie 2 (70 punktów):] Długość napisu końcowego to co najwyżej $2\,000\,001$ znaków.
    \end{description}

    \Constraints

    \begin{description}
        \item[Limit czasu:] 1 s.
        \item[Limit pamięci:] 64 MB.
    \end{description}

\end{document}

