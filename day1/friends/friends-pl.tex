\documentclass{boi2014-pl}

\usepackage{enumitem}
\usepackage{todonotes}

\renewcommand{\DayNum}{1}
\renewcommand{\TaskCode}{friends}
\renewcommand{\TaskName}{Przyjaciele}
\renewcommand{\TaskVersion}{1.3}

\begin{document}
    \begin{wrapfigure}{r}{3cm}
		\includegraphics[width=3cm]{\TaskCode.jpeg}
	\end{wrapfigure}
    Troje przyjaciół gra w następującą grę.
    Pierwszy z nich układa pewien napis $S$.
    Następnie drugi tworzy napis $T$ złożony z dwóch identycznych kopii napisu $S$.
    Na koniec, trzeci z przyjaciół dokłada jedną literę na początku, na końcu, bądź w środku napisu $T$, tworząc napis $U$.

    \Task
    Masz dany napis końcowy $U$. Zrekonstruuj napis $S$ ułożony przez pierwszego z przyjaciół.

    \Input
    W pierwszym wierszu wejścia znajduje się długość $N$ końcowego napisu $U$.
    W drugim wierszu znajduje się napis $U$.
    Napis składa się z wielkich liter alfabetu angielskiego (A, B, C, \ldots{}, Z).

    \Output
    Twój program powinien wypisać początkowy napis $S$.
    Są jednak dwa wyjątki:
    \begin{enumerate}
        \item Jeśli nie jest możliwe, by w wyniku zabawy z pewnego napisu początkowego $S$ powstał podany napis końcowy $U$, program powinien wypisać {\tt NOT POSSIBLE}
          (co oznacza: \emph{niemożliwe}).
        \item Jeśli początkowy napis $S$ można zrekonstruować na więcej niż jeden sposób, program powinien wypisać {\tt NOT
          UNIQUE} (co oznacza: \emph{niejednoznaczne}).
    \end{enumerate}
    

    \Examples

    \simpleexample{7\newline ABXCABC}{ABC}{}
    \simpleexample{6\newline ABCDEF}{NOT POSSIBLE}{}
    \simpleexample{9\newline ABABABABA}{NOT UNIQUE}{}

    \Scoring

    \begin{description}
        \item[Podzadanie 1 (35 punktów):] $2 \le N \le 2\,001$.
        \item[Podzadanie 2 (65 punktów):] $2 \le N \le 2\,000\,001$.
    \end{description}

    \Constraints

    \begin{description}
        \item[Limit czasu:] 0,5 s.
        \item[Dostępna pamięć:] 256 MB.
    \end{description}

\end{document}

