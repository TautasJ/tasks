\documentclass{boi2014-fi}

\renewcommand{\DayNum}{1}
\renewcommand{\TaskCode}{friends}
\renewcommand{\TaskName}{Kolme kaverusta}
\renewcommand{\TaskVersion}{1.3}

\begin{document}
    \begin{wrapfigure}{r}{4cm}
        \vspace{-24pt}
\includegraphics[width=4cm]{\TaskCode.jpeg}
\end{wrapfigure}
    Kolme kaverusta tykkäävät seuraavasta pelistä.
    Ensimmäinen valitsee merkkijonon $S$.
    Sitten toinen luo uuden merkkijonon $T$,
    joka muodostuu kahdesta merkkijonon $S$ kopiosta.
    Lopuksi kolmas lisää yhden kirjaimen merkkijonon $T$
    alkuun, loppuun tai johonkin keskelle ja
    muodostaa näin lopullisen merkkijonon $U$.

    \Task
    Sinulle on annettu lopullinen merkkijono $U$ ja tehtäväsi on
    muodostaa takaisin alkuperäinen merkkijono $S$.

    \Input
    Syötteen ensimmäisellä rivillä on lopullisen merkkijonon $U$ pituus $N$.
    Toisella rivillä on lopullinen merkkijono $U$. Se muodostuu $N$:stä
    englannin kielen suuresta kirjaimesta (A, B, C, \ldots{}, Z).

    \Output
    Ohjelmasi tulee tulostaa alkuperäinen merkkijono $S$.
    Kuitenkin on kaksi poikkeusta:
    \begin{enumerate}
        \item Jos ei ole mahdollista, että lopullinen merkkijono $U$ on luotu
        yllä olevalla tavalla, tulosta {\tt NOT POSSIBLE}.
        \item Jos alkuperäinen merkkijono $S$ ei ole yksikäsitteinen,
        tulosta {\tt NOT UNIQUE}.
    \end{enumerate}

    \Examples

    \simpleexample{7\newline ABXCABC}{ABC}{}
    \simpleexample{6\newline ABCDEF}{NOT POSSIBLE}{}
    \simpleexample{9\newline ABABABABA}{NOT UNIQUE}{}

    \Scoring

    \begin{description}
        \item[Osatehtävä 1 (35 pistettä):] $2 \le N \le 2001$.
        \item[Osatehtävä 2 (65 pistettä):] $2 \le N \le 2000001$.
    \end{description}

    \Constraints

    \begin{description}
        \item[Aikaraja:] 0.5 s.
        \item[Muistiraja:] 256 MB.
    \end{description}

\end{document}