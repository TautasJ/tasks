\documentclass{boi2014-fi}

\usepackage{todonotes}

\renewcommand{\DayNum}{1}
\renewcommand{\TaskCode}{sequence}
\renewcommand{\TaskName}{Lukujono}

\begin{document}
    \begin{wrapfigure}[5]{r}{5cm}
        \vspace{-24pt}
\includegraphics[width=5cm]{\TaskCode.jpeg}
\end{wrapfigure}

    Aatu kirjoitti liitutaululle lukujonon, joka muodostuu $K$ peräkkäisestä
    positiivisesta kokonaisluvusta luvusta $N$ aloittaen.
    Hänen lähdettyään Pekka tuli paikalle ja poisti jokaisesta luvusta
    kaikki numerot yhtä lukuun ottamatta.
    Näin syntyi lukujono, joka koostuu $K$ numerosta.

    \Task

    Annettuna on liitutaululle jäänyt lukujono,
    ja tehtäväsi on etsiä pienin mahdollinen arvo $N$,
    josta alkuperäinen lukujono on voinut alkaa.

    \Input

    Syötteen ensimmäisellä rivillä on yksi kokonaisluku $K$:
    lukujonon pituus. Toinen rivi sisältää $K$ kokonaislukua $B_1$, $B_2$, \dots,
    $B_K$: Pekan lukujono ($0 \le B_i \le 9$) siinä järjestyksessä,
    kun se on kirjoitettu taululle.

    \Output

    Tulosteessa tulee olle yksi rivi, jossa on pienin mahdollinen arvo $N$,
    josta alkuperäinen lukujono on voinut alkaa.

    \Example

    \example
    {
        6\newline
        7 8 9 5 1 2
    }
    {
        47
    }
    {
        $N = 47$ vastaisi Aatun lukujonoa
        $47\ 48\ 49\ 50\ 51\ 52$, josta Pekan lukujono todellakin saadaan.
        Koska mikään pienempi arvo $N$ ei toimisi, vastaus on 47.
    }

\Scoring

\begin{description}
    \item[Osatehtävä 1 (9 pistettä).] $1 \le K \le 1000$, oikea
        vastaus ei ole suurempi kuin $1000$.
    \item[Osatehtävä 2 (33 pistettä).] $1 \le K \le 1000$.
    \item[Osatehtävä 3 (25 pistettä).] $1 \le K \le 100\,000$, kaikki lukujonon jäsenet ovat samat.
    \item[Osatehtävä 4 (33 pistettä).] $1 \le K \le 100\,000$.
\end{description}

\Constraints

Aikaraja: 1 s.

Muistiraja: 256 MB.

\end{document}