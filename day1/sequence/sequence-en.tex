\documentclass{boi2014}

\usepackage{todonotes}

\renewcommand{\DayNum}{1}
\renewcommand{\TaskCode}{sequence}
\renewcommand{\TaskName}{Sequence}

\newcommand{\param}[1]{{\tt #1}}
\newcommand{\method}[1]{{\tt #1}}
\newcommand{\constant}[1]{{\tt #1}}

\begin{document}
    Adam wrote down on a blackboard a sequence of \param{K} consequtive
    positive integers starting with \param{N}. When he left, Billy came
    in and erased all but one digit from each number, thus creating a
    sequence of \param{K} integers between \constant{0} and \constant{9}.

    \Task

    Given the final sequence left on the blackboard, find the least
    possible value of \param{N} with which it could have occured.

    \Implementation

    Write a function \method{recreate\_sequence(K, B)} that takes
    the following parameters:
    \begin{itemize}
        \item \param{K} --- the length of either sequence
        \item \param{B} --- a one--dimensional array that describes
                            Billy's sequence, in the order in which
                            it is written on the blackboard: $\param{B}[i]$
                            ($0 \le i \le \param{K}-1$) is a digit
                            of $\param{N}+i$
    \end{itemize}

    Function \method{recreate\_sequence} has to return the least possible
    value of \param{N} with which this sequence could have occured.

    \Example

    Let us consider the example where
    \begin{figure}[H]
        \centering
        \constant{%
            \param{K} = 6
            \;
            \param{B} =%
            \begin{tabular}{c}
                7\\8\\9\\5\\1\\2
            \end{tabular}
        }
    \end{figure}

    Then setting $\param{N}=47$ would correspond to Adam's sequence
    being \constant{47 48 49 50 51 52} from which Billy's sequence
    can indeed be obtained. As no smaller value of \param{N}
    would work, your function has to return \constant{47}.

\Scoring

\begin{description}
    \item[Subtask 1 (? points).] $1 \le \param{K} \le 10$
    \item[Subtask 2 (? points).] $1 \le \param{K} \le 1000$, correct
        answer does not exceed $1000$
    \item[Subtask 3 (? points).] $1 \le \param{K} \le 1000$
    \item[Subtask 4 (? points).] $1 \le \param{K} \le 100\,000$
\end{description}

\Constraints

Time limit: $?$ s.

Memory limit: $?$ MB.

\end{document}

