\documentclass{boi2014}

\usepackage{todonotes}

\renewcommand{\DayNum}{1}
\renewcommand{\TaskCode}{sequence}
\renewcommand{\TaskName}{Sequence}

\begin{document}
    \begin{wrapfigure}{r}{5cm}
		\includegraphics[width=5cm]{\TaskCode.jpeg}
	\end{wrapfigure}

    Adam wrote down a sequence of $K$ consecutive positive integers starting
    with $N$ on a blackboard. When he left, Billy came in and erased all but one
    digit from each number, thus creating a sequence of $K$ integers between 0
    and 9.

    \Task

    Given the final sequence left on the blackboard, find the smallest
    value of $N$ with which it could have occured.

    \Input

    The first line of the input contains a single integer $K$ --- the length of
    the sequence. The second line contains $K$ integers $B_1$, $B_2$, \dots,
    $B_K$ --- Billy's sequence, in the order in which it is written on the
    blackboard.

    \Output

    The output should consist of a single line with the smallest value of
    $N$ with which this sequence could have occured.

    \Example

    \example
    {
        6\newline
        7 8 9 5 1 2
    }
    {
        47
    }
    {
        $N = 47$ would correspond to Adam's sequence
        being $<47\ 48\ 49\ 50\ 51\ 52>$ from which Billy's sequence
        can indeed be obtained. As no smaller value of $N$
        would work, the answer is 47.
    }

\Scoring

\begin{description}
    \item[Subtask 1 (? points).] $1 \le K \le 1000$, correct
        answer does not exceed $1000$
    \item[Subtask 2 (? points).] $1 \le K \le 1000$
    \item[Subtask 3 (? points).] $1 \le K \le 100\,000$, all
		elements of the given sequence are equal
    \item[Subtask 4 (? points).] $1 \le K \le 100\,000$
\end{description}

\Constraints

Time limit: $?$ s.

Memory limit: $?$ MB.

\end{document}

