\documentclass{boi2014}

\usepackage{todonotes}

\renewcommand{\DayNum}{1}
\renewcommand{\TaskCode}{sequence}
\renewcommand{\TaskName}{Sequence}

\newcommand{\param}[1]{{\tt #1}}
\newcommand{\method}[1]{{\tt #1}}
\newcommand{\constant}[1]{{\tt #1}}

\begin{document}
    Adam wrote down on a blackboard a sequence of \param{K} consequtive
    positive integers starting with \param{N}. When he left, Billy came
    in and erased all but one digit from each number, thus creating a
    sequence of \param{K} integers between \constant{0} and \constant{9}.

    \Task

    Given the final sequence left on the blackboard, find the smallest
    value of \param{N} with which it could have occured.

    \Input

    The first line of the input contains a single integer \param{K} ---
    the length of either sequence. The second line contains \param{K}
    space-separated integers $\param{B_1}$, $\param{B_2}$, \dots, $\param{B_K}$
    --- Billy's sequence, in the order in which it is written on the blackboard.

    \Output

    The output should consist of a single line with the smallest value of
    \param{N} with which this sequence could have occured.

    \Example

    \example
    {
        6\newline
        7 8 9 5 1 2
    }
    {
        47
    }
    {
        Setting $\param{N}=47$ would correspond to Adam's sequence
        being \constant{47 48 49 50 51 52} from which Billy's sequence
        can indeed be obtained. As no smaller value of \param{N}
        would work, the answer is \constant{47}.
    }

\Scoring

\begin{description}
    \item[Subtask 1 (? points).] $1 \le \param{K} \le 10$
    \item[Subtask 2 (? points).] $1 \le \param{K} \le 1000$, correct
        answer does not exceed $1000$
    \item[Subtask 3 (? points).] $1 \le \param{K} \le 1000$
    \item[Subtask 4 (? points).] $1 \le \param{K} \le 100\,000$
\end{description}

\Constraints

Time limit: $?$ s.

Memory limit: $?$ MB.

\end{document}

