\documentclass{boi2014-de}

\usepackage{todonotes}

\renewcommand{\DayNum}{1}
\renewcommand{\TaskCode}{sequence}
\renewcommand{\TaskName}{Folge}

\begin{document}
    \begin{wrapfigure}[5]{r}{5cm}
        \vspace{-24pt}
		\includegraphics[width=5cm]{\TaskCode.jpeg}
	\end{wrapfigure}

    Adam hat eine Folge von $K$ aufeinanderfolgenden positiven ganzen Zahlen, beginnend bei $N\ge1$, auf eine Tafel geschrieben.
    Nachdem er gegangen war, hat Billy von jeder Zahl alle bis auf eine Ziffer weg gewischt,
    sodass nun eine Folge von $K$ Ziffern zwischen $0$ und $9$ an der Tafel stehen.

    \Task

    Gegeben die Ziffernfolge die am Ende auf der Tafel steht, finde die kleinste Zahl $N$, mit der Adam die ursprüngliche Folge begonnen haben könnte.

    \Input

    Die erste Zeile der Eingabe enthält Zahl $K$, die Länge der Folge.
    Die zweite Zeile enthält $K$ ganze Zahlen $B_1$, $B_2$, \dots, $B_K$, die Ziffernfolge, die Billy hinterlassen hat.

    \Output

    Die Ausgabe soll eine Zeile mit der kleinsten Zahl $N$ enthalten, mit der Adam seine Folge begonnen haben könnte.

    \Example

    \example
    {
        6\newline
        7 8 9 5 1 2
    }
    {
        47
    }
    {
        $N = 47$ heißt, dass Adams Folge $47\ 48\ 49\ 50\ 51\ 52$ war, von der aus Billy durch passendes Wegwischen die Ziffernfolge kreiren kann. Da kein kleinerer Wert für $N$ funktioniert, ist die Antwort $47$.
    }

\Scoring

\begin{description}
    \item[Subtask 1 (9 points).] $1 \le K \le 1000$ und die korrekte Antwort ist nicht größer als $1000$.
    \item[Subtask 2 (33 points).] $1 \le K \le 1000$
    \item[Subtask 3 (25 points).] $1 \le K \le 100\,000$ und alle Elemente der Ziffernfolge sind gleich.
    \item[Subtask 4 (33 points).] $1 \le K \le 100\,000$
\end{description}

\Constraints

Zeitlimit: $1$ s.

Speicherlimit: $256$ MB.

\end{document}

