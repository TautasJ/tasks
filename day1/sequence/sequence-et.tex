\documentclass{boi2014-et}

\renewcommand{\DayNum}{1}
\renewcommand{\TaskCode}{sequence}
\renewcommand{\TaskName}{Jada}

\begin{document}

    \begin{wrapfigure}[5]{r}{5cm}
        \vspace{-24pt}
        \includegraphics[width=5cm]{\TaskCode.jpeg}
    \end{wrapfigure}

    Adam kirjutas tahvlile $K$ järjestikust positiivset täisarvu alates arvust $N$.
    Kui ta lahkus, tuli Billy ja kustutas tahvlilt numbreid nii, et igast arvust jäi järele üks number.
    Seega oli lõpuks tahvlil $N$ numbrist koosnev jada.

    \Task

    Leida tahvlile jäänud jada põhjal vähim võimalik $N$, millega esialgne jada alata võis.

    \Input

    Sisendi esimesel real on üks täisarv $K$: jada pikkus.
    Teisel real on $K$ täisarvu $B_1$, $B_2$, \dots, $B_K$ ($0 \le B_i \le 9$):
    Billy saadud jada elemendid selles järjekorras, nagu nad tahvlil olid.

    \Output

    Ainsale reale väljastada üks täisarv: vähim võimalik $N$, millest alustades
    oleks võimalik saada sisendis antud lõpptulemus.

    \Example

    \example
    {
        6\newline
        7 8 9 5 1 2
    }
    {
        47
    }
    {
        $N = 47$ korral oleks Adami esialgne jada $47\ 48\ 49\ 50\ 51\ 52$
        ja sellest võiks tõesti Billy jada saada. Kuna ühegi väiksema
        $N$ korral see nii pole, peabki vastus olema 47.
    }

    \Scoring

    \begin{description}
        \item[Alamülesanne 1 (9 punkti).] $1 \le K \le 1\,000$,
            vastus ei ületa $1\,000$.
        \item[Alamülesanne 2 (33 punkti).] $1 \le K \le 1\,000$.
        \item[Alamülesanne 3 (25 punkti).] $1 \le K \le 100\,000$,
            kõik Billy jada elemendid on ühesugused.
        \item[Alamülesanne 4 (33 punkti).] $1 \le K \le 100\,000$.
    \end{description}

    \Constraints

    \begin{description}
        \item[Ajalimiit:] 1 s.
        \item[Mälulimiit:] 256 MB.
    \end{description}

\end{document}
