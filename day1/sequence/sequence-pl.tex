\documentclass{boi2014-pl}

\usepackage{todonotes}

\renewcommand{\DayNum}{1}
\renewcommand{\TaskCode}{sequence}
\renewcommand{\TaskName}{Ciąg}

\begin{document}
    \begin{wrapfigure}[5]{r}{5cm}
        \vspace{-24pt}
		\includegraphics[width=5cm]{\TaskCode.jpeg}
	\end{wrapfigure}

    Adrian napisał na tablicy ciąg złożony z $K$ kolejnych, dodatnich liczb całkowitych, zaczynając od liczby $N$.
    Kiedy nikt nie patrzył, Bartek zmazał z każdej liczby zmazał wszystkie cyfry, oprócz jednej.
    W ten sposób utworzył ciąg $K$ liczb, pomiędzy 0 a 9.

    \Task

    Mając dany końcowy ciąg, znajdź najmniejszą wartość $N$, dla której mógł on powstać.

    \Input

    Pierwszy wiersz wejścia zawiera jedną liczbę całkowitą $K$ --- długość ciągu na tablicy.
    Drugi wiersz zawiera $K$ cyfr $B_1,\, B_2 ,\, \ldots ,\, B_K$ --- ciąg Bartka, w kolejności w jakiej jest na tablicy.
    
    \Output

    Wyjście powinno zawierać pojedynczy wiersz z najmniejszą wartością $N$, dla której taki ciąg mógł się pojawić.
    
    \Example

    \example
    {
        6\newline
        7 8 9 5 1 2
    }
    {
        47
    }
    {
        $N = 47$ odpowiada początkowemu ciągowi $<47\ 48\ 49\ 50\ 51\ 52>$, z którego mógł powstać ciąg Bartka. Jako że nie mogło tak być dla żadnego mniejszego $N$, to poprawna odpowiedź wynosi 47.
    }

\Scoring

\begin{description}
    \item[Podzadanie 1 (10 punktów).] $1 \le K \le 1\,000$, poprawna odpowiedź nie przekracza $1\,000$
    \item[Podzadanie 2 (30 punktów).] $1 \le K \le 1\,000$
    \item[Podzadanie 3 (25 punktów).] $1 \le K \le 100\,000$, wszystkie elementy w ciągu Bartka są równe
    \item[Podzadanie 4 (35 punktów).] $1 \le K \le 100\,000$
\end{description}

\Constraints

Limit czasu: $?$ s.

Limit pamięci: $?$ MB.

\end{document}

