\documentclass{boi2014-se}

\usepackage{todonotes}

\renewcommand{\DayNum}{1}
\renewcommand{\TaskCode}{sequence}
\renewcommand{\TaskName}{Sifferserier}

\begin{document}
    \begin{wrapfigure}[5]{r}{5cm}
        \vspace{-24pt}
		\includegraphics[width=5cm]{\TaskCode.jpeg}
	\end{wrapfigure}

    Adam skrev ner en serie av $K$ på varandra följande positiva heltal på en
    griffeltavla, startande från $N$.
    När han gått därifrån kom Billy in och suddade ut allt utom en siffra
    från varje tal, och skapade därigenom en ny serie av $K$ siffror.

    \Task

    Givet den slutgiltiga sifferserien på tavlan, hitta det minsta värdet på $N$
    med vilket den ursprungliga serien möjligen kunnat starta.

    \Input

    Första raden i indata innehåller ett enda heltal $K$ --- längden på serien.
    Andra raden innehåller $K$ heltal $B_1$, $B_2$, \dots, $B_K$ --- siffrorna
    i Billys serie ($0 \le B_i \le 9$), i samma ordning som de förekommer på tavlan.

    \Output

    Utdata ska innehålla en enda rad med det minsta möjliga värdet på $N$ som
    skulle kunnat resulterat i serien.

    \Example

    \example
    {
        6\newline
        7 8 9 5 1 2
    }
    {
        47
    }
    {
        $N = 47$ motsvarar att Adams serie är
        $<47\ 48\ 49\ 50\ 51\ 52>$ från vilken Billys serie är möjlig att skapa.
        Inget mindre $N$ fungerar, så svaret är 47.
    }

\Scoring

\begin{description}
    \item[Deluppgift 1 (9 poäng).] $1 \le K \le 1000$, korrekt
        svar överstiger ej $1000$.
    \item[Deluppgift 2 (33 poäng).] $1 \le K \le 1000$.
    \item[Deluppgift 3 (25 poäng).] $1 \le K \le 100\,000$, alla
        siffror i den givna serien är lika.
    \item[Deluppgift 4 (33 poäng).] $1 \le K \le 100\,000$.
\end{description}

\Constraints

Tidsgräns: 1 s.

Minnesgräns: 256 MB.

\end{document}

