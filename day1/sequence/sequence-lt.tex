\documentclass{boi2014-lt}

\usepackage{todonotes}

\renewcommand{\DayNum}{1}
\renewcommand{\TaskCode}{sequence}
\renewcommand{\TaskName}{Seka}

\begin{document}
    \begin{wrapfigure}[5]{r}{5cm}
        \vspace{-24pt}
		\includegraphics[width=5cm]{\TaskCode.jpeg}
	\end{wrapfigure}

    Ant lentos Adomas užrašė $K$ iš eilės einančių sveikųjų skaičių pradedant
    nuo $N$. Adomui išėjus, atėjo Bilas, kuris kiekvienam ant lentos užrašytam
    skaičiui nutrynė visus jo skaitmenis išskyrus vieną. Tokiu būdu jis gavo
    $K$ skaitmenų seką.

    \Task
    Duotai galutinei skaičių sekai lentoje, raskite mažiausią galimą skaičių $N$,
    kuriuo galėjo prasidėti Adomo užrašyta seka.

    \Input
    Pirmoje eilutė įrašytas vienas sveikasis skaičius $K$ --- sekos ilgis. Antroje
    eilutėje pateikti $K$ sveikųjų skaičių
    $B_1$, $B_2$, \dots, $B_K$ ($0 \le B_i \le 9$) --- skaičių seka, kurią ant
    lentos paliko Bilas.

    \Output
    Jūsų sprendimas turėtų išvesti mažiausią skaičių $N$, kuriuo galėjo prasidėti
    Adomo užrašyta seka.

    \Example

    \example
    {
        6\newline
        7 8 9 5 1 2
    }
    {
        47
    }
    {
        $N = 47$ atitinka Adomo skaičių seką $47\ 48\ 49\ 50\ 51\ 52$, iš kurios
        galima gauti Bilo skaičių seką. Kadangi mažesnių skaičių $N$, iš kurių
        galima suformuoti Bilo seką, nėra, tai atsakymas yra $47$.
    }

    \Scoring

    \begin{description}
        \item[Dalinė užduotis nr. 1 (9 taškai).] $1 \le K \le 1000$, teisingas atsakymas neviršyja $1000$
        \item[Dalinė užduotis nr. 2 (33 taškai).] $1 \le K \le 1000$
        \item[Dalinė užduotis nr. 3 (25 taškai).] $1 \le K \le 100\,000$, visi duotos sekos nariai yra lygūs
        \item[Dalinė užduotis nr. 4 (33 taškai).] $1 \le K \le 100\,000$
    \end{description}

    \Constraints

    \begin{description}
        \item[Laiko limitas:] 1 s.
        \item[Atminties limitas:] 256 MB.
    \end{description}

\end{document}

