\documentclass{boi2014-pl}

\usepackage{enumitem}
\usepackage{todonotes}
\usepackage{wrapfig}

\renewcommand{\DayNum}{2}
\renewcommand{\TaskCode}{demarcation}
\renewcommand{\TaskName}{Demarkacja}

\newcommand{\constant}[1]{{\tt #1}}

\begin{document}
    \begin{wrapfigure}{r}{3cm}
        \vspace{-24pt}
		\includegraphics[width=3cm]{\TaskCode.jpeg}
	\end{wrapfigure}

    Przez długie lata Bajtocja była wyspą, na której w pokoju żyli poddani króla Bajtazara I.
  Jednak po jego nagłej śmierci dwaj królewscy synowie -- bliźniacy Bitoni i Bajtoni -- nie
  mogli dojść do porozumienia, który z~nich powinien objąć tron. Postanowili więc podzielić
  wyspę na dwie prowincje, którymi będą rządzić niezależnie.

  Na prostokątnej mapie Bajtocja ma kształt wielokąta o $n$ bokach, przy czym każdy bok jest równoległy
  do jednego z~boków mapy, a każde dwa kolejne boki są do siebie prostopadłe.
  Bitoni i Bajtoni chcą podzielić ten wielokąt na dwie figury przystające za pomocą
  jednego odcinka równoległego do któregoś boku mapy i zawartego w wielokącie.
  (Figury są przystające jeśli jedna z nich może być przekształcona w drugą za pomocą obrotów, przesunięć oraz symetrii.)
  Zarówno współrzędne wierzchołków wielokąta, jak i odcinka dzielącego są całkowitoliczbowe.

  Królewscy synowie poprosili Cię, abyś stwierdził, czy taki podział jest
  w ogóle możliwy.

    \Task

    Mając dany kształt wyspy, odpowiedz, czy może być ona podzielona za pomocą poziomego lub pionowego
    odcinka na dwa przystające kawałki.
    Jeśli podział istnieje, znajdź jeden odcinek, który go powoduje.

    \Input
	W pierwszym wierszu wejścia znajduje się jedna liczba całkowita $N$ -- liczba wierzchołków.
        $i$-ty z kolejnych $N$ wierszy zawiera pary liczb całkowitych $X_i$ oraz $Y_i$, które są współrzędnymi $i$-tego wierzchołka.
    %\todo{In what order are vertices given?}

	\Output
        Twój program powinien wypisać pojedynczy wiersz.
        Jeśli jest możliwy podział wyspy na przystające części za pomocą poziomego lub pionowego odcinka o
        końcach w punktach $(x_1,y_1)$ oraz $(x_2,y_2)$, wypisz cztery liczby całkowite $x_1$, $y_1$, $x_2$, $y_2$ oddzielne spacjami.
        Musi zachodzić $x_1 = x_2$ lub $y_1 = y_2$.
 
        Jeśli podział nie jest możliwy, wypisz pojedyncze słowo \constant{NO}.

    \clearpage

    \Examples
	\example
	{
		10 \newline
		0 0 \newline
		1 0 \newline
		1 1 \newline
		3 1 \newline
		3 5 \newline
		2 5 \newline
		2 3 \newline
		1 3 \newline
		1 2 \newline
		0 2
	}
	{
		1 2 3 2
	}
	{
        Zauważ, że w poniższym przykładzie jest więcej niż jedna poprawna odpowiedź.
	}

	\example
	{
		6 \newline
		0 0 \newline
		1 0 \newline
		1 1 \newline
		2 1 \newline
		2 2 \newline
		0 2
	}
	{
		NO
	}

    \Scoring

    \begin{description}
        \item[Podzadanie 1 (? punktów).] $4 \le N \le 200$
        \item[Podzadanie 2 (? punktów).] $4 \le N \le 4\, 000$
        \item[Podzadanie 3 (? punktów).] $4 \le N \le 100\, 000$
    \end{description}

    \Constraints

    \begin{description}
        \item[Limit czasu:] ? s.
        \item[Dostępna pamięć:] ? MB.
    \end{description}

\end{document}
