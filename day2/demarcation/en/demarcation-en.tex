\documentclass{../../../latex/boi2014}

\usepackage{enumitem}
\usepackage{todonotes}
\usepackage{wrapfig}

\renewcommand{\DayNum}{2}
\renewcommand{\TaskCode}{demarcation}
\renewcommand{\TaskName}{Demarcation}

\newcommand{\param}[1]{{\tt #1}}
\newcommand{\method}[1]{{\tt #1}}
\newcommand{\constant}[1]{{\tt #1}}

\newcommand{\N}{\param{N}}
\newcommand{\X}{\param{X}}
\newcommand{\Y}{\param{Y}}
\newcommand{\xa}{\param{x1}}
\newcommand{\ya}{\param{y1}}
\newcommand{\xb}{\param{x2}}
\newcommand{\yb}{\param{y2}}


\begin{document}

    For a long time the island of Bytopia was ruled by the fair king
    Byteasar. But after the sudden death
    of the king, his two sons -- twins Biteon and Byteon -- could
    not come to an agreement which one of them should ascend the throne.
    Therefore they decided to divide the island into two provinces to
    rule them independently.  
 
    On a map Byteotia is shaped as a polygon of \N{} vertices. Every
    side of the polygon is parallel to a side of the map, and every
    two consecutive sides are perpendicular to each other.  Biteon
    and Byteon want to divide the polygon into two congruent figures,
    using one line segment contained in the polygon and parallel to a
    side of the map.  (Two figures are congruent if one can be transformed
    into the other using a combination of reflections, rotations and
    translations.) Coordinates of the polygon vertices and the end points
    of the dividing segment are integers.  
 
    The king's sons asked you to verify whether such a division is
    possible.

    \Task

    Given the shape of the island, determine if it can be partitioned
    by a horizontal or vertical segment into two congruent pieces. If
    it can, find one such segment.

    \Implementation
    You need to implement the procedure \method{divide\_island(N, X, Y)}
    which takes the following parameters:
    \begin{itemize}
        \item \N{} --- the number of vertices in the polygonal outline of
            the island
        \item \X{} --- a one--dimensional array that contains the
            $x$--coordinates of the vertices
        \item \Y{} --- a one--dimensional array that contains the
            $y$--coordinates of the vertices
    \end{itemize}

    More specifically, $(\X[i], \Y[i])$, $0 \le i \le \N-1$, are
    coordinates of the $i$--th vertex of the polygon.
    %\todo{In what order are vertices given?}

    Procedure \method{divide\_island} has to communicate the answer
    by calling one of the two provided procedures:
    \begin{itemize}
        \item \method{division\_line(\xa, \ya, \xb, \yb)} --- call
            this procedure if it is possible to divide the island
            into congruent parts with a horizontal or vertical segment.
            Its parameters must describe such segment: 
            (\xa, \ya) and (\xb, \yb) must be the coordinates of its
            endpoints. Either $\xa = \xb$ or $\ya = \yb$ must hold
        \item \method{impossible} --- call this procedure with no
            parameters if a suitable division cannot be found
    \end{itemize}

    \Examples
    For the example
    \begin{figure}[H]
        \centering
        \tt
        \N{} = 10
        \;
        \X{} =%
        \begin{tabular}{c}
            0\\1\\1\\3\\3\\2\\2\\1\\1\\0
        \end{tabular}
        \;
        \Y{} =%
        \begin{tabular}{c}
            0\\0\\1\\1\\5\\5\\3\\3\\2\\2
        \end{tabular}
    \end{figure}
    \method{division\_line(1, 2, 3, 2)} should be called (this is
    not the only correct choices of parameters).

    On the other hand, for the example
    \begin{figure}[H]
        \centering
        \tt
        \N{} = 6
        \;
        \X{} =%
        \begin{tabular}{c}
            0\\1\\1\\2\\2\\0
        \end{tabular}
        \Y{} =%
        \begin{tabular}{c}
            0\\0\\1\\1\\2\\2
        \end{tabular}
    \end{figure}
    the only correct solution is to call \method{impossible}.
    \todo{Need illustrations for examples}
    \Scoring

    \begin{description}
        \item[Subtask 1 (40 points).] $4 \le \N{} \le 200$
        \item[Subtask 2 (10 points).] $4 \le \N{} \le 5000$
        \item[Subtask 3 (50 points).] $4 \le \N{} \le 100\,000$
    \end{description}

    \Constraints

    \begin{description}
        \item[Time limit:] ? s.
        \item[Memory limit:] 128 MB.
    \end{description}

\end{document}
