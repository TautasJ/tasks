\documentclass{boi2014-lt}

\usepackage{enumitem}
\usepackage{todonotes}
\usepackage{wrapfig}

\renewcommand{\DayNum}{2}
\renewcommand{\TaskCode}{demarcation}
\renewcommand{\TaskName}{Demarkacija}

\newcommand{\constant}[1]{{\tt #1}}

\begin{document}
    \begin{wrapfigure}{r}{3cm}
        \vspace{-24pt}
		\includegraphics[width=3cm]{\TaskCode.jpeg}
	\end{wrapfigure}

    Karalius Baitazaras ilgai ir teisingai valdė Baitopijos salą. Tačiau po
    staigios mirties jo sūnūs dvyniai --- Baitonas ir Bitonas --- niekaip
    negalėjo sutarti, kuris iš jų turėtų paveldėti sostą. Jie nusprendė
    padalinti salą į dvi provincijas, kad galėtų jas valdyti nepriklausomai.
 
    Stačiakampyje žemėlapyje Baitopija yra $N$ kraštinių turintis daugiakampis.
    Visos daugiakampio kraštinės yra lygiagrečios žemėlapio kraštinėms, o
    kiekviena gretima daugiakampio kraštinė yra statmena viena kitai. Baitonas ir
    Bitonas norėtų padalinti salą į dvi kongruenčias figūras viena atkarpa, kuri
    būtų salos daugiakampio viduje ir būtų lygiagreti žemėlapio kraštinei. (Dvi
    figūros yra kongruenčios jeigu viena iš jų gali būti transformuota į kitą
    kokia nors atspindžio, posūkio arba poslinkio operacijų kombinacija.) Visų
    daugiakampio viršūnių ir ieškomos dalinančios atkarpos galų koordinatės yra
    sveikieji skaičiai.
 
    Karaliaus sūnūs prašo jūsų patikrinti, ar toks salos padalinimas yra
    įmanomas.

    \Task
    Duotai salos formai nustatykite, ar ji gali būti padalinta horizontalia arba
    vertikalia atkarpa į dvi kongruenčias figūras. Jeigu toks padalinimas
    egzistuoja, raskite salą dalinančią atkarpą.

    \Input
    Pirmoje eilutėje įrašytas vienas sveikasis skaičius $N$ -- daugiakampio
    viršūnių skaičius. Toliau kiekvienoje iš $N$ eilučių įrašyta tarpais atskirta
    sveikųjų skaičių pora $X_i$ ir $Y_i$ -- $i$-osios viršūnės koordinatės.

    \Output
    Jūsų programa turėtų išvesti vieną eilutę. Jeigu įmanoma padalinti salą į dvi
    kongruenčias figūras horizontalia arba vertikalia atkarpa, kurios galų
    koordinatės yra $(x_1, y_1)$ ir $(x_2, y_2)$, išspausdinkite $4$ tarpais
    atskirtus sveikuosius skaičius $x_1$, $y_1$, $x_2$ ir $y_2$. Jiems turėtų
    galioti bent viena iš lygybių $x_1 = y_1$ arba $y_1 = y_2$.

    Jeigu neįmanoma rasti tinkamo salos padalinimo, išveskite vieną žodį
    \constant{NO}.

    \Examples
	\example
	{
		10 \newline
		0 0 \newline
		1 0 \newline
		1 1 \newline
		3 1 \newline
		3 5 \newline
		2 5 \newline
		2 3 \newline
		1 3 \newline
		1 2 \newline
		0 2
	}
	{
		1 2 3 2
	}
	{
		Atkreipkite dėmesį, kad tai ne vienintelis galimas teisingas atsakymas.
	}

	\example
	{
		6 \newline
		0 0 \newline
		1 0 \newline
		1 1 \newline
		2 1 \newline
		2 2 \newline
		0 2
	}
	{
		NO
	}

    \Scoring

    \begin{description}
        \item[Dalinė užduotis nr. 1 (? taškų).] $4 \le N \le 200$
        \item[Dalinė užduotis nr. 2 (? taškų).] $4 \le N \le 4\ 000$
        \item[Dalinė užduotis nr. 3 (? taškų).] $4 \le N \le 100\ 000$
    \end{description}

    \Constraints

    \begin{description}
        \item[Laiko limitas:] ? s.
        \item[Atminties limitas:] ? MB.
    \end{description}

\end{document}
