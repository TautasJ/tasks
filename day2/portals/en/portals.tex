\documentclass{../../../latex/boi2014}

\usepackage{todonotes}

\renewcommand{\taskday}{2}
\renewcommand{\taskcode}{portals}
\renewcommand{\taskname}{Portals}

\newcommand{\param}[1]{{\tt #1}}
\newcommand{\method}[1]{{\tt #1}}
\newcommand{\constant}[1]{{\tt #1}}

\begin{document}
    There is a cake placed inside a labyrinth and you want
    to eat it. You have acquired a portal gun from
    Aperture Science\texttrademark, and you will use it to reach
    the cake in the shortest possible time. You
    have a map of the labyrinth, which is a grid with \param{H}
    rows and \param{W} columns. Each grid cell contains one
    of the following characters:

    \todo{Some indentation would be nice. Perhaps define a new environment?}
    \begin{description}
        \item[\constant{\#}] --- walled off square
        \item[\constant{.}] --- open square.
    \end{description}

    You may only walk on the open squares. Additionally, the rectangular
    area depicted on the map is surrounded by walls from the outside
    with no open spaces.

    The portal gun can at any time fire a portal in one of the four
    directions \emph{up}, \emph{left}, \emph{down} and \emph{right}.
    When you fire a portal in a direction, it will fly in that direction
    until it reaches the first wall. When this happens, a portal
    will be spawned on the side of that wall that faces you.

    At most two portals can exist at any time. If two portals
    are already placed in the labyrinth, then one of them (selected
    by you) will be removed immediately upon using the portal gun
    again. Firing a portal at a wall where there is already a portal
    placed will replace that portal (there may be at most one portal
    per side of wall). Notice that there may be multiple portals placed
    on different sides of same wall.

    \todo{Did I get this note right?}

    When two portals are placed on the map, you can use them to
    teleport yourself. When standing next to one of the portals,
    you can walk into it and end up at the square next to the other
    portal. Doing this takes as much time as moving between two
    adjacent squares.

    You may assume that firing portals does not take time and moving
    between two squares (or teleporting through portals) takes one unit
    of time.

    \Task

    Given the map of the labyrinth together with your starting location
    and the location of the cake, calculate how much time you need to
    reach the cake.

    \Implementation

    Write a function \method{least\_time(H, W, M, X, Y, U, V)} that takes
    the following parameters:
    \begin{itemize}
        \item \param{H} --- the number of rows in the map
        \item \param{W} --- the number of columns in the map
        \item \param{M} --- a two--dimensional array that describes the map.
                            Each entry \param{M}$[i][j]$
                            ($0 \le i \le \param{H}-1$,
                            $0 \le j \le \param{W}-1$) is the cell in
                            the $i$--th row and $j$--th column of the map.
                            It is
                            \begin{description}
                                \item[\constant{.}] if the corresponding
                                    place is open
                                \item[\constant{\#}] if there is a wall.
                            \end{description}
        \item \param{X} --- row of your starting location
        \item \param{Y} --- column of your starting location
        \item \param{U} --- row of the cake's location
        \item \param{V} --- column of the cake's location
    \end{itemize}

    Your starting location and location of the cake are positions
    on the map
    (that is, $0 \le \param{X}, \param{U} \le \param{H}-1$,
    $0 \le \param{Y}, \param{V} \le \param{W}-1$.)

    It is guaranteed that you can reach the cake from your starting
    location. In particular, locations $(\param{X}, \param{Y})$
    and $(\param{U}, \param{V})$ are open spaces (that is, $\param{M}
    [\param{X}][\param{Y}] = \param{M}[\param{U}][\param{V}] =
    \constant{.}$)

    \todo{Can $(X, Y) = (U, V)$?}

    Your function has to return the least time needed to reach the cake
    from the starting position.

    \Example

    Let us consider the example where

    \begin{figure}[H]
        \centering
        \constant{%
        \begin{tabular}{cc}
            \begin{tabular}{c}
                H=3 \\
                W=4 \\
                X=0 \\
                Y=3 \\
                U=2 \\
                V=0 
            \end{tabular}
            & M =
            \begin{tabular}{cccc}
                . & \# & . & . \\
                . & \# & . & \# \\
                . & . & . & . \\
                . & . & . & .
            \end{tabular}
        \end{tabular}
        }
    \end{figure}

\todo{
    Perhaps add a diagram showing the wall--boundary surrounding the
    labyrinth as well as positions of the cake and the player.
}

Here you have to reach the top--right corner from the bottom--left one.
One quickest sequence of moves is as follows:

\begin{itemize}
    \item move two squares right
    \item shoot one portal up, and one portal down
    \item move through the bottom portal --- you will appear at 
        the location $row = 0, column = 2$
    \item move one square right and reach the cake.
\end{itemize}

This takes time $4$ and no quicker way is possible. Therefore
your function should return \constant{4}.

\Subtasks
\todo{
    Looks ugly but I'm confident this layout can be very
    nice with sufficient tweeking.
}
\begin{figure}[H]
    \centering
    \begin{tabular}{cc}
        \begin{minipage}{0.4\textwidth}    
            \textbf{Subtask 1 (25 points)}
            \begin{description}
                \item $0 \le \param{H} \le 2$
                \item $0 \le \param{W} \le 2$
            \end{description}
        \end{minipage}
            &
        \begin{minipage}{0.4\textwidth}    
            \textbf{Subtask 2 (25 points)}
            \begin{description}
                \item $0 \le \param{H} \le 10$
                \item $0 \le \param{W} \le 10$
            \end{description}
        \end{minipage}
            \\ & \\
        \begin{minipage}{0.4\textwidth}    
            \textbf{Subtask 3 (25 points)}
            \begin{description}
                \item $0 \le \param{H} \le 100$
                \item $0 \le \param{W} \le 100$
            \end{description}
        \end{minipage}
            &
        \begin{minipage}{0.4\textwidth}    
            \textbf{Subtask 4 (25 points)}
            \begin{description}
                \item $0 \le \param{H} \le 1000$
                \item $0 \le \param{W} \le 1000$
            \end{description}
        \end{minipage}
    \end{tabular}
\end{figure}

\Details
\todo{
    Is this a good title?
}

Time limit: $1s$.

Memory limit: $256 MB$.

\todo{Also, boring stuff about graders that I don't have yet.}

\end{document}

