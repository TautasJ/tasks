Solution 1 (9 points):

We try all possible N values (starting from smallest - 1).
With each N, we check if it is the answer using brute-force - checking if all sequence numbers contains required digits.

Time complexity: O(N*K).

N - correct answer.

Since K <= 1000 and N <= 1000 in the first subtask, this solution is enough for it. 



Solution 2 (25 points):

We are assuming that all elements of the given sequence are equal (subtask 3).
Lets donate this digit as d.

If 1 <= d <= 8, than our answer N = d*10^x = d0..0.

If d = 9, than N = 8..89 (some number of digits '8' and one digit '9' at the end).

If d = 0, than N = 10^x = 10..0.

The length of N depends on K.
For example, if K = 5 and our sequence is '3 3 3 3 3', than d = 3 and N = d0..0 = 30..0 = 30.
If K = 11 and d = 3, than N = d0..0 = 30..0 = 300.

Time complexity: O(1).



Solution 3 (100 points):

We'll solve our task by solving a more complex task first: for each i-th sequence element we'll declare sequence A_i.
A_i is the sequence of digits which i-th sequence element has to have.
For our initial task, A_i = {d_i}, where d_i - ...

Lets donate N = (X)y, where y = N mod 10 is the last digit and X = N div 10.
Now we have to try all possible y values (0, 1, ..., 9).
After we have locked y value with one of possible values, our sequence looks like this: (X)y, (X)y+1, ..., (X)8, (X)9, (X+1)0, (X+1)1 ..., (X+1)8, (X+1)9, (X+2)0, ...
After removing last digit, we get sequence X, ..., X, X+1, ..., X+1, X+2, ...

What digits does X has to have?
(X)y has A_1, so X must have A_1 \ {y}. (X)(y+1) has A_2, so X must have A_2 \ {y+1} and so on.

By merging these requirements, we can get a new sequence of sets B_1, B_2, ...
B_1 declares required digits for X, B_2 - for (X+1) and so on.
We repeat the same steps with our new sequence.
What happens when we repeat these steps?
We started with sequence length K, so after first step our new sequence length will be not bigger than [K/10]+1.
So after log K steps our sequence will be of length 1 or 2 - at this step we can produce the answer.

Time complexity: O(K*log K)



Solution 4 (42 points):

If the previous solution is programmed inefficiently, it is possible to acquire an O(K^2) solution.

Time complexity: O(K^2).
