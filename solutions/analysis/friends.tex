
Quadratic solution (35 points):

We try removing all of possible N letters, thus simulating all possible cases of string T. 
For each letter, we construct a new string T', which contains the final string with one letter deleted. 
We check if T' is of the form S'S', and add S' as one of posible answers.
Depending on how many different answers we found, we output the result. 

Complexity: For each of N letters we match strings of length (N-1)/2, thus complexity
is O(N^2).

Linear solution (100 points):

The key observation is to notice that the initial string S, could only be in two of the positions:
either U[1..(N-1)/2], if the letter was inserted after a first copy of inital string S,
or U[(N-1)/2 + 1 .. N] if the letter was inserted before a second copy of initial string S. 

We have to check both cases, to see if initial string S is valid. We match it symbol-by-symbol with
remaining substring of U, to see if they only differ by one symbol. 

Yet again, depending on how many different answers we found, we output the result.

Complexity: We match two strings of length (N-1)/2 two times, thus complexity is O(N).


