
We are given a graph G, and we want to split it in cycles, such that:
 -In each cycle, each vertex is visited only once (except for the first vertex of the cycle).
 -Each edge of the graph is in exactly one cycle. 

They key observation to solve the problem is that a graph can be split in such cycles if and only if
the degrees of all vertices are even. Knowing this we can find any cycle in graph, remove it from the graph
(operation does not change the parity of degrees of vertices), save the cycle as an answer, and continue
looking for cycles, while there are still edges left. 

This can be implemented in O(N + M) time with a modified Depth-first traversal algorithm:

We spread from initial vertex, until we find a visited vertex, which means, we found a cycle. We
traverse back, removing edges of the cycle. 

The only question we have left to answer is the following: how do we store edges?

Solution 1 (38 points):
We store a boolean adjenceny matrix.
Time and memory complexity: O(N^2).

Solution 2 (55 points):
We store adjecency lists and a set of pairs of integers, denoting edges as used. 
Time complexity O(M log M), Memory complexity O(N + M)

Solution 3 (100 points):
We store a list of edges e[0], e[1], ..., e[M], and adjecency lists for each vertex, 
linking to original list of edges. Now we can mark edges as removed in original list
of edges, thus checking if edge is removed in O(1)

Overall time and memory complexity: O(N + M).

