\documentclass{boi2014-lt}

\usepackage{enumitem}
\usepackage{todonotes}

\renewcommand{\DayNum}{0}
\renewcommand{\TaskCode}{numbers}
\renewcommand{\TaskName}{Numbers}

\begin{document}

    Duotai sveikųjų skaičių porai raskite visus tarp jų esančius sveikuosius
    skaičius.

    \Input
    Vienintelėje pradinių duomenų eilutėje pateikti du sveikieji skaičiai
    $A$ ir $B$.

    \Output
    Pirmoje eilutėje įrašykite vieną sveikąjį skaičių --- rastų sveikųjų skaičių,
    kurie yra tarp $A$ ir $B$, skaičių. Jeigu egzistuoja bent vienas toks
    skaičius, jie visi turėtų būti išvardinti didėjimo tvarka antroje eilutėje.
    
    \Examples

    \simpleexample
    {
        8 14
    }
    {
        5 \newline
        9 10 11 12 13
    }

    \simpleexample
    {
        1 2
    }
    {
        0
    }

    \Scoring
    Jūsų sprendimas gaus visus taškus už dalinę užduoti, jeigu pateiks teisingus
    sprendimus kiekvienam testui šioje dalinėje užduotyje. Kitu atveju, jeigu
    pirmoji rezultatų eilutė bus teisinga visiems testams, jūsų sprendimas gaus
    $50\%$ dalinės užduoties taškų. Visais kitais atvejais jūsų sprendimas negaus
    taškų. Taip pat sprendimas negaus taškų už dalinę užduotį, jeigu bent viename
    teste iš jos viršys leistina laiko limitą arba grąžins vykdymo klaidą.

    \begin{description}

        \item[Subtask 1 (30 points):] $1 \le A, B \le 1000$. 
        \item[Subtask 2 (70 points):] $1 \le A, B \le 10^{15}$,
            $A$ ir $B$ daugiausiai skirsis $100\,000$.
    \end{description}

    \Constraints

    \begin{description}
        \item[Time limit:] 1 s.
        \item[Memory limit:] 64 MB.
    \end{description}

\end{document}

